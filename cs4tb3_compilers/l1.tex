\documentclass[12pt]{article}

\usepackage[utf8]{inputenc}
\usepackage{synttree}

\title{CS 4TB3
\\\vspace{10mm}
\large \textbf{Assignment 1}
\vspace{40mm}
}
\author{
	Susie Yu \#000955758\\ 
	with\\
	Maxim Vasiliev \#400043983
}
\date{January 13 2017}

\begin{document}
\maketitle
\newpage
\tableofcontents
\newpage
\pagenumbering{arabic}

\section{Question 1}
\subsection{1a}
\synttree
[S[a][S[$\epsilon$]][b][S[a][S[$\epsilon$]][b][S[$\epsilon$]]]]

\hfill \break
\synttree
[S[a][S[b][S[$\epsilon$]][a][S[$\epsilon$]]][b][S[$\epsilon$]]]

\subsection{1b}
The language generated by G is any combination of "a"s and "b"s, such that the number of "a"s equals the number of "b"s.


\section{Question 2}
\subsection{2a}
S $\to$ S - S $|$ T \\
T $\to$ T + T $|$ N \\
N $\to$ a $|$ b $|$ c $|$ d \\

\synttree
[S [S [T [N [a]]]] [-] [S [S [T [T [N [b]]] [+] [T [N [c]]]]] [-] [S [T [N [d]]]]]]

\subsection{2b}
S $\to$ S + S $|$ T \\
T $\to$ T - T $|$ N \\
N $\to$ a $|$ b $|$ c $|$ d \\

\synttree
[S [S [T [T [N [a]]] [-] [T [ N [b]]]]] [+] [S [T [T [N [c]]] [-] [T [N [d]]]]]]

\subsection{2c}
S $\to$ S + T $|$ S - T $|$ T \\
T $\to$ a $|$ b $|$ c $|$ d \\

\synttree
[S [S [S [S [T [a]]] [-] [T [b]]] [+] [T [c]]] [-] [T [d]]]

\subsection{2d}
S $\to$ T + S $|$ T - S $|$ T \\
T $\to$ a $|$ b $|$ c $|$ d \\

\synttree
[S [T [a]][-][S [T [b]] [+] [S [T [c]] [-] [S [T [d]]]]]]


\section{Question 3}

First we prove the inclusion L(G) $\subseteq$ $\lbrace$ a\textsuperscript{n} b c\textsuperscript{n} $|$ n $\geq$ 0 $\rbrace$ .
\noindent \newline
Using production 1 gives us S $\Rightarrow$ A, which is a non-terminal so we continue. Using productions 1, then 2 gives us S $\Rightarrow$ A $\Rightarrow$ b = (a)\textsuperscript{0}(b)(c)\textsuperscript{0} , so the inclusion holds. Using production 1, then 3, then 2 we get S $\Rightarrow$ A $\Rightarrow$ aAc $\Rightarrow$ abc = a\textsuperscript{1}bc\textsuperscript{1} so the inclusion holds. For the recursive production rule 3, if we assume that A of the right side is of the form a\textsuperscript{n}bc\textsuperscript{n}, then aAc is a\textsuperscript{n+1}bc\textsuperscript{n+1}, hence A on the left side is again of the form a\textsuperscript{n} b c\textsuperscript{n}, for some n.Thus we have L(G) $\subseteq$ $\lbrace$ a\textsuperscript{n} b c\textsuperscript{n} $|$ n $\geq$ 0 $\rbrace$ 

\noindent \newline \newline 
Now we show that any a\textsuperscript{n}bc\textsuperscript{n} for n $\geq$ 0 can be generated by G. a\textsuperscript{0}bc\textsuperscript{0} = b can be generated by S $\Rightarrow$ A $\Rightarrow$ b. a\textsuperscript{1}bc\textsuperscript{1} = abc can be generated by S $\Rightarrow$ A $\Rightarrow$ aSc $\Rightarrow$ abc. Suppose we can generate a\textsuperscript{n}bc\textsuperscript{n}, then we must have S $\Rightarrow$* a\textsuperscript{n}bc\textsuperscript{n}, but since S only goes to A, we must then have A $\Rightarrow$* a\textsuperscript{n}bc\textsuperscript{n}. We need to show that a\textsuperscript{n+1}bc\textsuperscript{n+1} can be generated as well. Which is done by S $\Rightarrow$ A $\Rightarrow$ aAb $\Rightarrow$* aa\textsuperscript{n}bc\textsuperscript{n}c = a\textsuperscript{n+1}bc\textsuperscript{n+1}. Thus we have L(G) $\supseteq$ $\lbrace$ a\textsuperscript{n} b c\textsuperscript{n} $|$ n $\geq$ 0 $\rbrace$ 

\end{document}